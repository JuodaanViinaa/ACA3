\documentclass[a4paper,12pt]{article}
\usepackage[utf8]{inputenc}
\usepackage[T1]{fontenc}
\usepackage[spanish]{babel}
\usepackage{csquotes}
\usepackage{anysize}
\usepackage{graphicx}
\usepackage{hyperref}
%\usepackage{amsfonts}
%\usepackage{tikz}
%\usepackage{amsmath}
\marginsize{25mm}{25mm}{25mm}{25mm}

\title{Metodología y estadística}
\author{Daniel Maldonado}
\date{}

\begin{document}
{\scshape\bfseries \maketitle}

{\noindent\scshape\bfseries Diseños de investigación.}

Un diseño de investigación es un plan de trabajo usado para responder preguntas de la mejor manera posible. En análisis de la conducta utilizamos investigación cuantitativa, basada en la medición de variables del modo más objetivo posible. 

Existen diseños experimentales y no experimentales.

\begin{itemize}
    \item No experimentales: No existe manipulación de variables, sino solamente medición y observación en los entornos naturales de los fenómenos.
        \begin{itemize}
            \item Transversal o transeccional: Se recolectan datos en un único momento.
            \item Longitudinal o evolutivo: Se analiza el cambio en las variables a través del tiempo. Ej., estudios de cohorte.
        \end{itemize}
    \item Experimentales: Consisten en la manipulación de variables independientes para obtener resultados medibles sobre variables dependientes.

        Permiten establecer confiablemente relaciones de causalidad.

        Requieren de:
        \begin{itemize}
            \item Manipulación intencional de una o más variables independientes.
            \item Medición de los cambios en las variables dependientes de manera válida y confiable.
            \item Control de la validez interna de las situaciones experimentales.
        \end{itemize}

        Se dividen en:
        \begin{itemize}
            \item Experimentos puros
            \begin{itemize}
                \item Manipulación intencional de variables independientes.
                \item Medición de variables dependientes
                \item Control y validez
                \item Dos o más grupos de comparación
                \item Asignación aleatoria o emparejamiento
            \end{itemize}
            \item Estudios preexperimentales: grado de control mínimo. Ejemplos: estudio de caso con una sola medición, diseño de pre-test post-test con un solo grupo.
            \item Cuasiexperimentos: Grupos intactos, como se encuentran en su entorno natural.
        \end{itemize}
\end{itemize}


\end{document}
