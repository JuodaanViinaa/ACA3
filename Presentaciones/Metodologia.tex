\documentclass[a4paper,12pt]{article}
\usepackage[utf8]{inputenc}
\usepackage[T1]{fontenc}
\usepackage[spanish]{babel}
\usepackage{csquotes}
\usepackage{anysize}
\usepackage{graphicx}
\usepackage{hyperref}
%\usepackage{amsfonts}
%\usepackage{tikz}
%\usepackage{amsmath}
\marginsize{25mm}{25mm}{25mm}{25mm}

\title{Metodología y estadística}
\author{Daniel Maldonado}
\date{}

\begin{document}
{\scshape\bfseries \maketitle}

{\noindent\scshape\bfseries Diseños de investigación.}

Un diseño de investigación es un plan de trabajo usado para responder preguntas de la mejor manera posible. En análisis de la conducta utilizamos investigación cuantitativa, basada en la medición de variables conductuales del modo más objetivo posible.

Existen diseños experimentales y no experimentales.

\begin{itemize}
    \item No experimentales: No existe manipulación de variables, sino solamente medición y observación en los entornos naturales de los fenómenos. Aunque hay menor control de variables, lo que limita la validez interna, existe una mayor validez externa.
        \begin{itemize}
            \item Transversal o transeccional: Se recolectan datos en un único momento.
            \item Longitudinal o evolutivo: Se analiza el cambio en las variables a través del tiempo. Ej., estudios de cohorte.
        \end{itemize}
    \item Experimentales: Consisten en la manipulación de variables independientes para obtener resultados medibles sobre variables dependientes.

        Permiten establecer confiablemente relaciones de causalidad.

        Requieren de:
        \begin{itemize}
            \item Manipulación intencional de una o más variables independientes.
            \item Medición de los cambios en las variables dependientes de manera válida y confiable.
            \item Control de la validez interna de las situaciones experimentales. Se garantiza que el efecto observado en las variables dependientes se debe a la manipulación de las independientes. Existen múltiples amenazas a la validez interna, por ejemplo:
                \begin{itemize}
                    \item Historia
                    \item Maduración
                    \item Fallos de instrumentación
                    \item Regresión a la media
                    \item Muerte experimental
                    \item Grupos no equivalentes
                \end{itemize}
        \end{itemize}

        Se dividen en:
        \begin{itemize}
            \item Experimentos puros
                \begin{itemize}
                    \item Manipulación intencional de variables independientes.
                    \item Medición de variables dependientes
                    \item Control y validez
                    \item Dos o más grupos de comparación
                    \item Asignación aleatoria o emparejamiento
                    \item Ejemplos:
                        \begin{itemize}
                            \item Diseño de post-prueba con grupo control
                            \item Diseño pre- y post-prueba con control
                            \item Diseños factoriales
                                \\[2mm]
                                \begin{tabular}{l|c|c|c|c}
                                                & Salina     & Haloperidol & Morfina        & Metilfenidato\\
                                    \hline
                                    Dosis Baja  & 10 Sujetos & 10 Sujetos  & 10 Sujetos     & 10 Sujetos\\
                                    \hline
                                    Dosis Media & 10 Sujetos & 10 Sujetos  & 10 Sujetos     & 10 Sujetos\\
                                    \hline
                                    Dosis Alta  & 10 Sujetos & 10 Sujetos  & 10 Sujetos     & 10 Sujetos\\
                                \end{tabular}

                        \end{itemize}
                \end{itemize}
            \item Estudios preexperimentales: grado de control mínimo. Ejemplos:
                \begin{itemize}
                    \item Estudio de caso con una sola medición
                    \item Diseño de pre-test post-test con un solo grupo
                \end{itemize}
            \item Cuasiexperimentos: Hay manipulación de variables, pero no se garantiza la equivalencia de grupos: grupos intactos, como se encuentran en su entorno natural.
        \end{itemize}
\end{itemize}

Los pasos generales para desarrollar un estudio/experimento son:

\begin{itemize}
    \item Leer todo lo posible sobre el tema de interés y desarrollar un marco teórico
    \item 
\end{itemize}


\end{document}
